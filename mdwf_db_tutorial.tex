\documentclass{article}
\usepackage[utf8]{inputenc}
\usepackage{geometry}
\usepackage{listings}
\usepackage{xcolor}
\usepackage{fancyvrb}
\usepackage{hyperref}
\usepackage{amsmath}
\usepackage{amsfonts}
\usepackage{amssymb}

\geometry{margin=1in}

% Define colors for syntax highlighting
\definecolor{codegreen}{rgb}{0,0.6,0}
\definecolor{codegray}{rgb}{0.5,0.5,0.5}
\definecolor{codepurple}{rgb}{0.58,0,0.82}
\definecolor{backcolour}{rgb}{0.95,0.95,0.92}

% Configure listings for terminal output
\lstdefinestyle{terminal}{
    backgroundcolor=\color{backcolour},
    commentstyle=\color{codegreen},
    keywordstyle=\color{magenta},
    numberstyle=\tiny\color{codegray},
    stringstyle=\color{codepurple},
    basicstyle=\ttfamily\footnotesize,
    breakatwhitespace=false,
    breaklines=true,
    captionpos=b,
    keepspaces=true,
    numbers=left,
    numbersep=5pt,
    showspaces=false,
    showstringspaces=false,
    showtabs=false,
    tabsize=2,
    literate={_}{\_}1
}

\lstset{style=terminal}

% Custom command for code with underscores
\newcommand{\code}[1]{\texttt{\detokenize{#1}}}

\title{MDWF Database Management Tool\\Complete Command Reference}
\author{}
\date{}

\begin{document}

\maketitle

\tableofcontents
\newpage

\section{Lattice QCD Parameters}

The MDWF database tracks Domain Wall Fermion lattice QCD ensembles with the following physics parameters:

\subsection{Required Parameters}

\begin{itemize}
\item \textbf{beta}: Gauge coupling parameter (e.g., 6.0)
\item \textbf{b}: Domain wall height parameter (e.g., 1.8)
\item \textbf{Ls}: Domain wall extent in 5th dimension (e.g., 24)
\item \textbf{mc}: Charm quark mass (e.g., 0.8555)
\item \textbf{ms}: Strange quark mass (e.g., 0.0725)
\item \textbf{ml}: Light quark mass (e.g., 0.02)
\item \textbf{L}: Spatial lattice size (e.g., 32)
\item \textbf{T}: Temporal lattice size (e.g., 64)
\end{itemize}

\subsection{Directory Structure}

Ensembles are organized in a hierarchical directory structure:

\begin{lstlisting}[language=bash, caption=Directory Structure]
TUNING/b6.0/b1.8Ls24/mc0.8555/ms0.0725/ml0.02/L32/T64/
ENSEMBLES/b6.0/b1.8Ls24/mc0.8555/ms0.0725/ml0.0195/L32/T64/
\end{lstlisting}

Each ensemble directory contains:
\begin{itemize}
\item \texttt{cnfg/}: Configuration files
\item \texttt{jlog/}: Job logs
\item \code{log_hmc/}: HMC logs
\item \texttt{slurm/}: SLURM scripts
\end{itemize}

\section{Database Management Commands}

\subsection{init-db: Initialize Database}

Initialize a new MDWF database and directory structure.

\subsubsection{Options}
\begin{itemize}
\item \code{--db-file DB_FILE}: Path to SQLite database (auto-discovered)
\item \code{--base-dir BASE_DIR}: Root directory for TUNING/ and ENSEMBLES/ (default: current)
\end{itemize}

\subsubsection{Example Usage}

\begin{lstlisting}[language=bash, caption=Initialize Database]
$ mdwf_db init-db
Ensured directory: /Users/wyatt/Development/mdwf_db
Ensured directory: /Users/wyatt/Development/mdwf_db/TUNING
Ensured directory: /Users/wyatt/Development/mdwf_db/ENSEMBLES
init_database returned: True

$ mdwf_db init-db --base-dir /scratch/lattice
Ensured directory: /scratch/lattice
Ensured directory: /scratch/lattice/TUNING
Ensured directory: /scratch/lattice/ENSEMBLES
init_database returned: True
\end{lstlisting}

\subsection{add-ensemble: Add New Ensemble}

Add a new ensemble to the database with physics parameters.

\subsubsection{Options}
\begin{itemize}
\item \code{--db-file DB_FILE}: Path to SQLite database (auto-discovered)
\item \code{-p PARAMS, --params PARAMS}: Space-separated key=val pairs (required)
\item \code{-s \{TUNING,PRODUCTION\}, --status}: Ensemble status (required)
\item \code{-d DIRECTORY, --directory}: Explicit directory path (optional)
\item \code{-b BASE_DIR, --base-dir}: Root directory (default: current)
\item \code{--description DESCRIPTION}: Free-form description (optional)
\end{itemize}

\subsubsection{Example Usage}

\begin{lstlisting}[language=bash, caption=Add Ensemble Examples]
# Basic TUNING ensemble
$ mdwf_db add-ensemble \
  -p "beta=6.0 b=1.8 Ls=24 mc=0.8555 ms=0.0725 ml=0.02 L=32 T=64" \
  -s TUNING
Ensemble added: ID=1

# PRODUCTION ensemble with description
$ mdwf_db add-ensemble \
  -p "beta=6.0 b=1.8 Ls=24 mc=0.8555 ms=0.0725 ml=0.0195 L=32 T=64" \
  -s PRODUCTION \
  --description "Production run with lighter quark masses"
Ensemble added: ID=2
Marked PRODUCTION in DB: OK

# Custom directory path
$ mdwf_db add-ensemble \
  -p "beta=6.0 b=1.8 Ls=24 mc=0.8555 ms=0.0725 ml=0.01 L=32 T=64" \
  -s TUNING \
  -d ./custom/path/to/ensemble
Ensemble added: ID=3

# With custom base directory
$ mdwf_db add-ensemble \
  -p "beta=6.0 b=1.8 Ls=24 mc=0.8555 ms=0.0725 ml=0.02 L=32 T=64" \
  -s TUNING \
  -b /scratch/lattice
Ensemble added: ID=4
\end{lstlisting}

\subsection{query: Query Ensemble Information}

List ensembles or show detailed information for a specific ensemble.

\subsubsection{Options}
\begin{itemize}
\item \code{--db-file DB_FILE}: Path to SQLite database (auto-discovered)
\item \code{-e ENSEMBLE, --ensemble}: Ensemble ID, path, or "." for current directory
\item \code{--detailed}: Show physics parameters and operation counts in list mode
\end{itemize}

\subsubsection{Example Usage}

\begin{lstlisting}[language=bash, caption=Query Examples]
# List all ensembles (basic)
$ mdwf_db query
[1] (PRODUCTION) /Users/wyatt/Development/mdwf_db/ENSEMBLES/b6.0/b1.8Ls24/mc0.8555/ms0.0725/ml0.02/L32/T64
[2] (PRODUCTION) /Users/wyatt/Development/mdwf_db/ENSEMBLES/b6.0/b1.8Ls24/mc0.8555/ms0.0725/ml0.0195/L32/T64

# List all ensembles with details
$ mdwf_db query --detailed
[1] (PRODUCTION) /Users/wyatt/Development/mdwf_db/ENSEMBLES/b6.0/b1.8Ls24/mc0.8555/ms0.0725/ml0.02/L32/T64
    Parameters: L=32, Ls=24, T=64, b=1.8, beta=6.0, mc=0.8555, ml=0.02, ms=0.0725
    Operations: 2
    Description: Example ensemble for documentation

[2] (PRODUCTION) /Users/wyatt/Development/mdwf_db/ENSEMBLES/b6.0/b1.8Ls24/mc0.8555/ms0.0725/ml0.0195/L32/T64
    Parameters: L=32, Ls=24, T=64, b=1.8, beta=6.0, mc=0.8555, ml=0.0195, ms=0.0725
    Operations: 0

# Show detailed info for specific ensemble
$ mdwf_db query -e 1
ID          = 1
Directory   = /Users/wyatt/Development/mdwf_db/ENSEMBLES/b6.0/b1.8Ls24/mc0.8555/ms0.0725/ml0.02/L32/T64
Status      = PRODUCTION
Created     = 2025-06-26T13:37:33.638950
Description = Example ensemble for documentation
Parameters:
    L = 32
    Ls = 24
    T = 64
    b = 1.8
    beta = 6.0
    mc = 0.8555
    ml = 0.02
    ms = 0.0725

=== Operation history ===
Op 1: HMC_TUNE [RUNNING]
  Created: 2025-06-26T13:38:04.948828
  Updated: 2025-06-26T13:38:04.948828
    config_end = 50
    config_start = 0
    slurm_job = 12345

Op 2: PROMOTE_ENSEMBLE [COMPLETED]
  Created: 2025-06-26T13:38:32.123456
  Updated: 2025-06-26T13:38:32.123456

# Query by directory path
$ mdwf_db query -e ./ENSEMBLES/b6.0/b1.8Ls24/mc0.8555/ms0.0725/ml0.02/L32/T64

# Query current directory (when inside ensemble)
$ cd ENSEMBLES/b6.0/b1.8Ls24/mc0.8555/ms0.0725/ml0.02/L32/T64
$ mdwf_db query -e .
\end{lstlisting}

\subsection{promote-ensemble: Promote to Production}

Move a TUNING ensemble to PRODUCTION status and directory.

\subsubsection{Options}
\begin{itemize}
\item \code{--db-file DB_FILE}: Path to SQLite database (auto-discovered)
\item \code{-e ENSEMBLE, --ensemble}: Ensemble ID, path, or "." (required)
\item \code{--base-dir BASE_DIR}: Root directory (default: current)
\item \code{--force}: Skip confirmation prompt
\end{itemize}

\subsubsection{Example Usage}

\begin{lstlisting}[language=bash, caption=Promote Ensemble Examples]
# Promote with confirmation
$ mdwf_db promote-ensemble -e 1
Promote ensemble 1:
  from /Users/wyatt/Development/mdwf_db/TUNING/b6.0/b1.8Ls24/mc0.8555/ms0.0725/ml0.02/L32/T64
    to /Users/wyatt/Development/mdwf_db/ENSEMBLES/b6.0/b1.8Ls24/mc0.8555/ms0.0725/ml0.02/L32/T64
Continue? [y/N]: y
Created operation 2: Created
Promotion OK

# Promote without confirmation
$ mdwf_db promote-ensemble -e 1 --force
Promote ensemble 1:
  from /Users/wyatt/Development/mdwf_db/TUNING/b6.0/b1.8Ls24/mc0.8555/ms0.0725/ml0.02/L32/T64
    to /Users/wyatt/Development/mdwf_db/ENSEMBLES/b6.0/b1.8Ls24/mc0.8555/ms0.0725/ml0.02/L32/T64
Created operation 2: Created
Promotion OK

# Promote by directory path
$ mdwf_db promote-ensemble -e ./TUNING/b6.0/b1.8Ls24/mc0.8555/ms0.0725/ml0.02/L32/T64

# Promote current directory
$ cd TUNING/b6.0/b1.8Ls24/mc0.8555/ms0.0725/ml0.02/L32/T64
$ mdwf_db promote-ensemble -e . --force
\end{lstlisting}

\subsection{clear-history: Clear Operation History}

Clear all operation history for an ensemble while preserving the ensemble record.

\subsubsection{Options}
\begin{itemize}
\item \code{--db-file DB_FILE}: Path to SQLite database (auto-discovered)
\item \code{-e ENSEMBLE, --ensemble}: Ensemble ID, path, or "." (required)
\item \code{--force}: Skip confirmation prompt
\end{itemize}

\subsubsection{Example Usage}

\begin{lstlisting}[language=bash, caption=Clear History Examples]
# Clear history with confirmation
$ mdwf_db clear-history -e 1
Clear all operation history for ensemble 1?
This will remove all operations but preserve the ensemble record.
Continue? [y/N]: y
Cleared 2 operations for ensemble 1

# Clear history without confirmation
$ mdwf_db clear-history -e 1 --force
Cleared 2 operations for ensemble 1

# Clear by directory path
$ mdwf_db clear-history -e ./ENSEMBLES/b6.0/b1.8Ls24/mc0.8555/ms0.0725/ml0.02/L32/T64

# Clear current directory
$ cd ENSEMBLES/b6.0/b1.8Ls24/mc0.8555/ms0.0725/ml0.02/L32/T64
$ mdwf_db clear-history -e . --force
\end{lstlisting}

\subsection{remove-ensemble: Remove Ensemble}

Remove an ensemble and all its operations from the database.

\subsubsection{Options}
\begin{itemize}
\item \code{--db-file DB_FILE}: Path to SQLite database (auto-discovered)
\item \code{-e ENSEMBLE, --ensemble}: Ensemble ID, path, or "." (required)
\item \code{--force}: Skip confirmation prompt
\item \code{--remove-directory}: Also delete the on-disk directory tree
\end{itemize}

\subsubsection{Example Usage}

\begin{lstlisting}[language=bash, caption=Remove Ensemble Examples]
# Remove from database only (preserve files)
$ mdwf_db remove-ensemble -e 1
Remove ensemble 1 from database?
This will delete all ensemble and operation records.
Continue? [y/N]: y
Removed ensemble 1 from database

# Remove database record and delete files
$ mdwf_db remove-ensemble -e 1 --remove-directory --force
Removed ensemble 1 from database
Deleted directory: /Users/wyatt/Development/mdwf_db/ENSEMBLES/b6.0/b1.8Ls24/mc0.8555/ms0.0725/ml0.02/L32/T64

# Remove by directory path
$ mdwf_db remove-ensemble -e ./ENSEMBLES/b6.0/b1.8Ls24/mc0.8555/ms0.0725/ml0.02/L32/T64

# Remove current ensemble
$ cd ENSEMBLES/b6.0/b1.8Ls24/mc0.8555/ms0.0725/ml0.02/L32/T64
$ mdwf_db remove-ensemble -e . --force
\end{lstlisting}

\section{Job Script Generation Commands}

\subsection{glu-input: Generate GLU Input Files}

Generate input files for the GLU gauge field utility program.

\subsubsection{Options}
\begin{itemize}
\item \code{--db-file DB_FILE}: Path to SQLite database (auto-discovered)
\item \code{-e ENSEMBLE_ID, --ensemble-id}: ID of the ensemble (required)
\item \code{-o OUTPUT_FILE, --output-file}: Path for GLU input file (required)
\item \code{-g GLU_PARAMS, --glu-params}: Space-separated key=val pairs (optional)
\item \code{-t \{smearing,gluon_props,other\}, --type}: Calculation type (default: smearing)
\end{itemize}

\subsubsection{GLU Parameters}
GLU parameters use flat names (no dots) and include:
\begin{itemize}
\item \texttt{CONFNO}: Configuration number (default: 24)
\item \code{DIM_0, DIM_1, DIM_2}: Spatial dimensions (auto-set from ensemble)
\item \code{DIM_3}: Temporal dimension (auto-set from ensemble)
\item \texttt{SMEARTYPE}: Smearing algorithm (default: STOUT)
\item \texttt{SMITERS}: Number of smearing iterations (default: 8)
\item \texttt{ALPHA1}: Primary smearing parameter (default: 0.75)
\item \texttt{ALPHA2}: Secondary smearing parameter (default: 0.4)
\item \texttt{ALPHA3}: Tertiary smearing parameter (default: 0.2)
\item \texttt{GFTYPE}: Gauge fixing type (default: COULOMB)
\item \code{GF_TUNE}: Gauge fixing tuning (default: 0.09)
\item \texttt{ACCURACY}: Gauge fixing accuracy (default: 14)
\item \code{MAX_ITERS}: Maximum iterations (default: 650)
\end{itemize}

\subsubsection{Example Usage}

\begin{lstlisting}[language=bash, caption=GLU Input Examples]
# Basic smearing input with defaults
$ mdwf_db glu-input -e 1 -o smear.in
Generated GLU input file: smear.in

# View generated file
$ cat smear.in
MODE = SMEARING    
HEADER = NERSC
    DIM_0 = 32
    DIM_1 = 32
    DIM_2 = 32
    DIM_3 = 64
CONFNO = 24
RANDOM_TRANSFORM = NO
SEED = 0
GFTYPE = COULOMB
    GF_TUNE = 0.09
    ACCURACY = 14
    MAX_ITERS = 650
CUTTYPE = GLUON_PROPS
FIELD_DEFINITION = LINEAR
    MOM_CUT = CYLINDER_CUT
    MAX_T = 7
    MAXMOM = 4
    CYL_WIDTH = 2.0
    ANGLE = 60
    OUTPUT = ./
SMEARTYPE = STOUT
    DIRECTION = ALL
    SMITERS = 8
    ALPHA1 = 0.75
    ALPHA2 = 0.4
    ALPHA3 = 0.2
U1_MEAS = U1_RECTANGLE
    U1_ALPHA = 0.07957753876221914
    U1_CHARGE = -1.0
CONFIG_INFO = 2+1DWF_b2.25_TEST
    STORAGE = CERN
BETA = 6.0
    ITERS = 1500
    MEASURE = 1
    OVER_ITERS = 4
    SAVE = 25
    THERM = 100

# Custom smearing parameters
$ mdwf_db glu-input -e 1 -o custom_smear.in \
  -g "CONFNO=168 SMITERS=50 ALPHA1=0.1"
Generated GLU input file: custom_smear.in

# Gauge fixing input
$ mdwf_db glu-input -e 1 -o gauge_fix.in -t other \
  -g "CONFNO=100 GFTYPE=LANDAU ACCURACY=16"
Generated GLU input file: gauge_fix.in

# Gluon properties calculation
$ mdwf_db glu-input -e 1 -o gluon_props.in -t gluon_props \
  -g "CONFNO=200 MAXMOM=6 MAX_T=10"
Generated GLU input file: gluon_props.in
\end{lstlisting}

\subsection{smear-script: Generate Smearing Scripts}

Generate complete SLURM scripts for configuration smearing using GLU.

\subsubsection{Options}
\begin{itemize}
\item \code{--db-file DB_FILE}: Path to SQLite database (auto-discovered)
\item \code{-e ENSEMBLE_ID, --ensemble-id}: ID of the ensemble (required)
\item \code{-j JOB_PARAMS, --job-params}: SLURM job parameters (required)
\item \code{-g GLU_PARAMS, --glu-params}: GLU smearing parameters (optional)
\item \code{-o OUTPUT_FILE, --output-file}: Output script path (auto-generated if not specified)
\end{itemize}

\subsubsection{Job Parameters}
Required job parameters:
\begin{itemize}
\item \code{mail_user}: Email address for job notifications
\item \code{config_start}: First configuration number to smear
\item \code{config_end}: Last configuration number to smear
\end{itemize}

Optional job parameters with defaults:
\begin{itemize}
\item \code{account}: SLURM account (default: m2986\_g)
\item \texttt{constraint}: Node constraint (default: gpu)
\item \texttt{queue}: SLURM partition (default: regular)
\item \code{time_limit}: Job time limit (default: 06:00:00)
\item \texttt{nodes}: Number of nodes (default: 1)
\item \code{cpus_per_task}: CPUs per task (default: 16)
\item \texttt{gpus}: GPUs per node (default: 4)
\item \code{gpu_bind}: GPU binding (default: none)
\item \texttt{ranks}: MPI ranks (default: 4)
\item \code{bind_sh}: CPU binding script (default: bind.sh)
\end{itemize}

\subsubsection{Example Usage}

\begin{lstlisting}[language=bash, caption=Smear Script Examples]
# Basic smearing job
$ mdwf_db smear-script -e 1 \
  -j "mail_user=user@example.com config_start=100 config_end=200"
Generated smearing script: glu_smear_STOUT8_100_200.sh

# Custom smearing and job parameters
$ mdwf_db smear-script -e 1 \
  -j "mail_user=user@example.com config_start=100 config_end=200 time_limit=12:00:00 nodes=2" \
  -g "SMITERS=10 ALPHA1=0.8 SMEARTYPE=APE"
Generated smearing script: glu_smear_APE10_100_200.sh

# Specify output file
$ mdwf_db smear-script -e 1 -o custom_smear.sh \
  -j "mail_user=user@example.com config_start=100 config_end=200"
Generated smearing script: custom_smear.sh

# High-precision smearing
$ mdwf_db smear-script -e 1 \
  -j "mail_user=user@example.com config_start=50 config_end=100 time_limit=24:00:00" \
  -g "SMITERS=20 ALPHA1=0.1 ALPHA2=0.05 ACCURACY=16"
Generated smearing script: glu_smear_STOUT20_50_100.sh

# Custom account and queue
$ mdwf_db smear-script -e 1 \
  -j "mail_user=user@example.com config_start=1 config_end=50 account=lattice_qcd queue=debug time_limit=02:00:00"
Generated smearing script: glu_smear_STOUT8_1_50.sh
\end{lstlisting}

\subsection{hmc-script: Generate HMC Scripts}

Generate HMC XML parameters and SLURM batch scripts for gauge configuration generation.

\subsubsection{Options}
\begin{itemize}
\item \code{--db-file DB_FILE}: Path to SQLite database (auto-discovered)
\item \code{-e ENSEMBLE_ID, --ensemble-id}: ID of the ensemble (required)
\item \code{-a ACCOUNT, --account}: SLURM account name (required)
\item \code{-m \{tepid,continue,reseed\}, --mode}: HMC run mode (required)
\item \code{--base-dir BASE_DIR}: Root directory (default: current)
\item \code{-x XML_PARAMS, --xml-params}: HMC XML parameters (optional)
\item \code{-j JOB_PARAMS, --job-params}: SLURM job parameters (optional)
\item \code{-o OUTPUT_FILE, --output-file}: Output script path (auto-generated if not specified)
\end{itemize}

\subsubsection{HMC Modes}
\begin{itemize}
\item \texttt{tepid}: Initial thermalization run (TepidStart)
\item \texttt{continue}: Continue from existing checkpoint (CheckpointStart)
\item \texttt{reseed}: Start new run with different seed (CheckpointStartReseed)
\end{itemize}

\subsubsection{Job Parameters}
Required job parameter:
\begin{itemize}
\item \code{cfg_max}: Maximum configuration number to generate
\end{itemize}

Optional job parameters with defaults:
\begin{itemize}
\item \texttt{constraint}: Node constraint (default: gpu)
\item \code{time_limit}: Job time limit (default: 17:00:00)
\item \code{cpus_per_task}: CPUs per task (default: 32)
\item \texttt{nodes}: Number of nodes (default: 1)
\item \code{gpus_per_task}: GPUs per task (default: 1)
\item \code{gpu_bind}: GPU binding (default: none)
\item \code{mail_user}: Email notifications (from environment)
\item \texttt{queue}: SLURM partition (default: regular)
\item \code{exec_path}: Path to HMC executable (auto-detected)
\item \code{bind_script}: CPU binding script (auto-detected)
\end{itemize}

\subsubsection{XML Parameters}
Available HMC XML parameters:
\begin{itemize}
\item \texttt{StartTrajectory}: Starting trajectory number (default: 0)
\item \texttt{Trajectories}: Number of trajectories to generate (default: 50)
\item \texttt{MetropolisTest}: Perform Metropolis test (true/false, default: true)
\item \texttt{NoMetropolisUntil}: Trajectory to start Metropolis (default: 0)
\item \texttt{PerformRandomShift}: Perform random shift (true/false, default: true)
\item \texttt{StartingType}: Start type (auto-set by mode)
\item \texttt{Seed}: Random seed (for reseed mode)
\item \texttt{MDsteps}: Number of MD steps (default: 2)
\item \texttt{trajL}: Trajectory length (default: 1.0)
\end{itemize}

\subsubsection{Example Usage}

\begin{lstlisting}[language=bash, caption=HMC Script Examples]
# Basic HMC script for new ensemble
$ mdwf_db hmc-script -e 1 -a m2986 -m tepid -j "cfg_max=100"
Generated HMC script: hmc_tepid.sh
Wrote HMCparameters.xml to /Users/wyatt/Development/mdwf_db/ENSEMBLES/b6.0/b1.8Ls24/mc0.8555/ms0.0725/ml0.02/L32/T64

# Continue existing run
$ mdwf_db hmc-script -e 1 -a m2986 -m continue \
  -j "cfg_max=200 time_limit=24:00:00" \
  -x "StartTrajectory=100 Trajectories=100"
Generated HMC script: hmc_continue.sh

# Custom parameters and output file
$ mdwf_db hmc-script -e 1 -a m2986 -m tepid -o custom_hmc.sh \
  -j "cfg_max=50 nodes=2 time_limit=12:00:00 mail_user=user@example.com" \
  -x "MDsteps=4 trajL=0.75 Seed=12345"
Generated HMC script: custom_hmc.sh

# Reseed mode with custom seed
$ mdwf_db hmc-script -e 1 -a lattice_qcd -m reseed \
  -j "cfg_max=150 constraint=gpu time_limit=20:00:00" \
  -x "Seed=98765 StartTrajectory=100 Trajectories=50"
Generated HMC script: hmc_reseed.sh

# Debug run with short time limit
$ mdwf_db hmc-script -e 1 -a m2986 -m tepid \
  -j "cfg_max=10 time_limit=01:00:00 queue=debug" \
  -x "Trajectories=10 MetropolisTest=false"
Generated HMC script: hmc_tepid.sh
\end{lstlisting}

\subsection{hmc-xml: Generate HMC XML Files}

Generate standalone HMC parameters XML files for an ensemble.

\subsubsection{Options}
\begin{itemize}
\item \code{--db-file DB_FILE}: Path to SQLite database (auto-discovered)
\item \code{-e ENSEMBLE_ID, --ensemble-id}: ID of the ensemble (required)
\item \code{-m \{tepid,continue,reseed\}, --mode}: Run mode (required)
\item \code{-b BASE_DIR, --base-dir}: Root directory (default: current)
\item \code{-x XML_PARAMS, --xml-params}: XML parameter overrides (optional)
\end{itemize}

\subsubsection{Example Usage}

\begin{lstlisting}[language=bash, caption=HMC XML Examples]
# Basic XML generation
$ mdwf_db hmc-xml -e 1 -m tepid
Wrote HMCparameters.xml to /Users/wyatt/Development/mdwf_db/ENSEMBLES/b6.0/b1.8Ls24/mc0.8555/ms0.0725/ml0.02/L32/T64

# View generated XML
$ cat ENSEMBLES/b6.0/b1.8Ls24/mc0.8555/ms0.0725/ml0.02/L32/T64/HMCparameters.xml
<?xml version="1.0" ?>
<grid>
  <HMCparameters>
    <StartTrajectory>0</StartTrajectory>
    <Trajectories>50</Trajectories>
    <MetropolisTest>false</MetropolisTest>
    <NoMetropolisUntil>0</NoMetropolisUntil>
    <PerformRandomShift>false</PerformRandomShift>
    <StartingType>TepidStart</StartingType>
    <Seed>973655</Seed>
    <MD>
      <name>
        <elem>OMF2_5StepV</elem>
        <elem>OMF2_5StepV</elem>
        <elem>OMF4_11StepV</elem>
      </name>
      <MDsteps>2</MDsteps>
      <trajL>1.0</trajL>
    </MD>
  </HMCparameters>
</grid>

# Custom XML parameters
$ mdwf_db hmc-xml -e 1 -m continue \
  -x "StartTrajectory=100 Trajectories=100 MetropolisTest=true"
Wrote HMCparameters.xml to /Users/wyatt/Development/mdwf_db/ENSEMBLES/b6.0/b1.8Ls24/mc0.8555/ms0.0725/ml0.02/L32/T64

# Reseed mode with custom seed
$ mdwf_db hmc-xml -e 1 -m reseed \
  -x "Seed=42 StartTrajectory=50 Trajectories=25"
Wrote HMCparameters.xml to /Users/wyatt/Development/mdwf_db/ENSEMBLES/b6.0/b1.8Ls24/mc0.8555/ms0.0725/ml0.02/L32/T64

# Custom base directory
$ mdwf_db hmc-xml -e 1 -m tepid -b /scratch/lattice \
  -x "Trajectories=200 MDsteps=4"
Wrote HMCparameters.xml to /scratch/lattice/ENSEMBLES/b6.0/b1.8Ls24/mc0.8555/ms0.0725/ml0.02/L32/T64
\end{lstlisting}

\section{Operation Tracking Commands}

\subsection{update: Record Operations}

Create or update operation records in the database for tracking job execution.

\subsubsection{Options}
\begin{itemize}
\item \code{--db-file DB_FILE}: Path to SQLite database (auto-discovered)
\item \code{-e ENSEMBLE_ID, --ensemble-id}: ID of the ensemble (required)
\item \code{-o OPERATION_TYPE, --operation-type}: Type of operation (required)
\item \code{-s \{RUNNING,COMPLETED,FAILED\}, --status}: Operation status (required)
\item \code{-i OPERATION_ID, --operation-id}: ID of existing operation to update (optional)
\item \code{-p PARAMS, --params}: Operation parameters (optional)
\end{itemize}

\subsubsection{Operation Types}
Common operation types:
\begin{itemize}
\item \code{HMC_TUNE}: HMC tuning run
\item \code{HMC_PRODUCTION}: HMC production run
\item \code{GLU_SMEAR}: Configuration smearing
\item \code{PROMOTE_ENSEMBLE}: Ensemble promotion
\end{itemize}

\subsubsection{Operation Parameters}
Common parameters:
\begin{itemize}
\item \code{config_start}: First configuration number
\item \code{config_end}: Last configuration number
\item \code{exit_code}: Job exit code
\item \texttt{runtime}: Job runtime in seconds
\item \code{slurm_job}: SLURM job ID
\item \texttt{host}: Execution hostname
\end{itemize}

\subsubsection{Example Usage}

\begin{lstlisting}[language=bash, caption=Update Operation Examples]
# Record new HMC operation
$ mdwf_db update -e 1 -o HMC_TUNE -s RUNNING \
  -p "config_start=0 config_end=100 slurm_job=12345"
Created operation 1: Created

# Update operation to completed
$ mdwf_db update -e 1 -o HMC_TUNE -s COMPLETED -i 1 \
  -p "config_start=0 config_end=100 exit_code=0 runtime=3600"
Updated operation 1: Updated

# Record failed operation
$ mdwf_db update -e 1 -o GLU_SMEAR -s FAILED \
  -p "config_start=100 config_end=200 exit_code=1 slurm_job=12346"
Created operation 2: Created

# Record smearing operation
$ mdwf_db update -e 1 -o GLU_SMEAR -s RUNNING \
  -p "config_start=100 config_end=200 slurm_job=12347 host=gpu-node-01"
Created operation 3: Created

# Update with runtime information
$ mdwf_db update -e 1 -o GLU_SMEAR -s COMPLETED -i 3 \
  -p "config_start=100 config_end=200 exit_code=0 runtime=7200"
Updated operation 3: Updated
\end{lstlisting}

\section{Workflow Examples}

\subsection{Complete Ensemble Workflow}

This section demonstrates a complete workflow from initialization to production.

\begin{lstlisting}[language=bash, caption=Complete Workflow Example]
# 1. Initialize database
$ mdwf_db init-db
Ensured directory: /Users/wyatt/Development/mdwf_db
Ensured directory: /Users/wyatt/Development/mdwf_db/TUNING
Ensured directory: /Users/wyatt/Development/mdwf_db/ENSEMBLES
init_database returned: True

# 2. Add tuning ensemble
$ mdwf_db add-ensemble \
  -p "beta=6.0 b=1.8 Ls=24 mc=0.8555 ms=0.0725 ml=0.02 L=32 T=64" \
  -s TUNING \
  --description "Tuning run for new parameters"
Ensemble added: ID=1

# 3. Generate HMC script for thermalization
$ mdwf_db hmc-script -e 1 -a m2986 -m tepid -j "cfg_max=100"
Generated HMC script: hmc_tepid.sh

# 4. Record HMC operation start
$ mdwf_db update -e 1 -o HMC_TUNE -s RUNNING \
  -p "config_start=0 config_end=100 slurm_job=12345"
Created operation 1: Created

# 5. Update operation when completed
$ mdwf_db update -e 1 -o HMC_TUNE -s COMPLETED -i 1 \
  -p "exit_code=0 runtime=14400"
Updated operation 1: Updated

# 6. Generate smearing script
$ mdwf_db smear-script -e 1 \
  -j "mail_user=user@example.com config_start=50 config_end=100"
Generated smearing script: glu_smear_STOUT8_50_100.sh

# 7. Record smearing operation
$ mdwf_db update -e 1 -o GLU_SMEAR -s RUNNING \
  -p "config_start=50 config_end=100 slurm_job=12346"
Created operation 2: Created

# 8. Check ensemble status
$ mdwf_db query -e 1
ID          = 1
Directory   = /Users/wyatt/Development/mdwf_db/TUNING/b6.0/b1.8Ls24/mc0.8555/ms0.0725/ml0.02/L32/T64
Status      = TUNING
Parameters:
    L = 32, Ls = 24, T = 64, b = 1.8, beta = 6.0
    mc = 0.8555, ml = 0.02, ms = 0.0725
=== Operation history ===
Op 1: HMC_TUNE [COMPLETED]
Op 2: GLU_SMEAR [RUNNING]

# 9. Promote to production
$ mdwf_db promote-ensemble -e 1 --force
Promote ensemble 1:
  from /Users/wyatt/Development/mdwf_db/TUNING/b6.0/b1.8Ls24/mc0.8555/ms0.0725/ml0.02/L32/T64
    to /Users/wyatt/Development/mdwf_db/ENSEMBLES/b6.0/b1.8Ls24/mc0.8555/ms0.0725/ml0.02/L32/T64
Created operation 3: Created
Promotion OK

# 10. Continue production run
$ mdwf_db hmc-script -e 1 -a m2986 -m continue \
  -j "cfg_max=500" -x "StartTrajectory=100 Trajectories=400"
Generated HMC script: hmc_continue.sh
\end{lstlisting}

\end{document} 